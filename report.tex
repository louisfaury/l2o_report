%!TEX encoding = IsoLatin

%% Document is article 
\documentclass[a4paper]{report}

%% ----------------------------------------------------- PACKAGES ----------------------------------------------------- %%
\usepackage{coolReport}
\usepackage{tabularx}
\usepackage{breakcites}
\usepackage{subcaption}
\usepackage{slashbox}

\graphicspath{{./img/}}
\cRTitle{Master Project Report}


\newcommand\blankpage{%
    \null
    \thispagestyle{empty}%
    \addtocounter{page}{-1}%
    \newpage}
    
%% ---------------------------------------------------------TITLE --------------------------------------------------------- %%

%% ---------------------------------------------------- DOCUMENT ---------------------------------------------------- %%

\begin{document}
	\include{./title/title}
	
	\abstract
	{
		Learning to optimize - the idea that we can learn from data algorithms that optimize a numerical criterion - has recently been at the heart of a growing number of research efforts. It is mostly motivated by two observations: first, most optimization methods used by machine learning practitioners are built on top of human understanding of high-dimensional loss-landscapes, for which human intuition are limited. Also, optimization constitutes today a major bottleneck for the deployment of machine learning methods, namely because of the need for hyper-parameter tuning. If learning to optimize over one problem instance is relatively easy, a major challenge of this approach is that we want to learn a policy that is able to optimize over classes of functions that are different from the classes that the policy was trained on.
		\\
		
		Machine learning is at the core of the technology developed by Criteo. The complexification of the models used in production require that engineers spend more and more time tuning their models beforehand. The development of a general-use, hyper-parameter-free optimizer is therefore of great interest for Criteo. The following report present the work I have completed on this subject while interning in the Deep Learning research team in Criteo's Paris offices, under the supervision of Flavian Vasile. 
		\\
		
		We propose a novel way of framing learning to optimize as a problem of learning a good navigation policy on a partially observable loss surface. To this end, we develop Rover Descent, a solution that allows us to learn a broad optimization policy from training only on a small set of prototypical two-dimensional surfaces that encompasses classically hard cases such as valleys, plateaus, cliffs and saddles and by using strictly zeroth-order information. We show that, without having access to gradient or curvature information, we achieve fast convergence on optimization problems not presented at training time, such as the Rosenbrock function and other two dimensional hard functions. We extend our framework to optimize over high dimensional functions and show good preliminary results.
	} 
		
	\tableofcontents
	
	\newpage
	\begingroup
	\let\clearpage\relax
		\listoffigures
	\endgroup

	\chapter{Introduction}
	{
		\section{Criteo}
		{
		
			\paragraph{} Criteo is one of the most successful entrepreneurship adventures of the last few years, especially in the web technologies field. Funded in 2005 by Jean-Baptiste Rudelle, Franck le Ouay and Romain Niccoli, the company offers personalized ad-retargeting solutions. In only 3 years, the company becomes the world's leader for that business, with a presence in more that 30 countries in 3 continents. With nearly 3000 employees, the company serves hundred of billions of ads yearly, for a transaction total of 500 billions of dollars in 2016. 
			
			The main concern of Criteo is to bring back e-commerce costumers back to the retailer website so that they can concretize an ongoing order, a purchase new items. Criteo targets such users and proposes new product that fits their center of interests. Various techniques are employed to evaluate a given user's interests, its susceptibility of clicking on and ad and of converting to a sale on the advertiser website. 
			
			Criteo is the intermediary between two key actors: on the one hand, advertisers which signal the presence of quality users on their website, and who which to see them return. These involves a various member of e-commerce brands (La Redoute, Les 3 Suisses, ..), who trust Criteo to advertise for their product. On the other hand, publishers monetize their advertisement inventory. When a user loads a page on their website, they propose ad-placements on the page, usually through a bidding system. They involve large media website, or real-time bidding platforms like Google RTB. 
			
			\paragraph{} For obvious privacy and capacity reasons, Criteo can't store explicit information for every user that visit an advertiser website. Also, relationships between products (that could be advertised for) is extremely hard to engineer. Machine learning is a natural solution to these two problems, and is therefore at the very core of the technology proposed by Criteo. A wide variety of machine learning models is deployed at Criteo, and therefore a great extent of money and efforts are put in the training of these models. Also, pushed by the global enthusiasm around deep learning and the record breaking capacities it has shown, Criteo is pushing for its deployment in production, for many of its core technologies. 

		}

		\section{Motivations}
		{
			Numerical optimization is at the heart of most modern machine learning techniques, and often constitutes an important bottleneck when it comes to their deployment. Some of these techniques fall in mastered and well-known instances of numerical optimization problems (\emph{e.g} convex optimization, see \cite{nocedal2006numerical}) for which strong theoretical results have paved the way for the development of efficient new algorithms for many years. Recently, the advent of more complex models such a deep artificial neural networks led to the creation of several methods targeting high-dimensional, non-convex problems (the most famous ones being momentum \cite{nesterov1983method}, Adadelta \cite{zeiler2012adadelta} and Adam \cite{kingma2014adam}), now used as black-box algorithms by a majority of practitioners. Other attempts in this field use some additional problem-specific structure, like the work by \cite{martens2010deep} that leverages fast multiplication by the Hessian to yield better performing optimization policies, though computationally demanding. A common point to all these algorithms is that they leverage human-based understanding of loss surfaces, and usually require tuning hyper-parameters to achieve state-of-the-art performance. This tuning process can sometimes reveal mysterious behavior of the handled optimizers, making it reserved to human experts or the subject of a long and tedious search. Also, the process results in a static optimizer which excels at the specific task, but is likely to perform poorly on others. 
	
	\paragraph{}
	If the limitations of hand-designed algorithms come from poor human understanding of high-dimensional loss landscapes, it is natural to ask what machine learning can do for the design of optimization algorithms. Recently, \cite{andrychowicz2016learning} and \cite{li2016learning} both introduced two frameworks for learning optimization algorithms. While the former proposes to learn task-specific optimizers, the latter aims to produce task-independent optimization policies. While in the most general case this is bound to fail - as suggested by the No Free Lunch theorem for combinatorial optimization \cite{wolpert1997no} - we also believe that data driven techniques can be robust on a great variety of problems. \cite{li2016learning} manages to learn meta-learner that optimize a variety of neural network based losses (different datasets, different architecture), but fail to generalize to other problems. 
	
	\paragraph{} With this internship, we aimed at tackling the problem of meta-generalization and learn an optimizer that produces satisfying result on large classes of loss functions. It should display two main characteristics: an ability to navigate high-dimensional landscapes it has never seen during its meta-training, and also a lack of hyper-parameters which will enable it to be used truly as a black-box optimizer (and bypass the tedious hyper-parameter tuning phase).
		}
	}
	
	\section{Proposed approach and contributions}
	{
		Most optimization algorithms can be framed, for a given objective function $f$ and a current iterate $\theta_i$, as the problem of finding an appropriate update $\Delta \theta_i$. This update can for instance depend on past gradient information, rescaled gradient using curvature information or many other features. In a general manner, we can write $\Delta \theta_i = \phi( \theta_i, h(f,\theta_{i-1},\hdots,\theta_0),\xi)$ where $h(\cdot)$ denotes the set of features accumulated during the optimization procedure, and $\xi$ denotes the optimization hyper-parameters. 
		In our approach, we aim to bypass computing gradient and curvature information and learn the optimization features directly from data. This should allows us to obtain local state descriptors that can outperform classical features in terms of generalization on unseen loss functions and input data distributions. In this vein, we draw an analogy between learning an optimization algorithm and learning a navigation policy while having access to raw local observations of the landscape, which is also the inspiration for the name of our method, \emph{Rover Descent}. 
	Our algorithm contains three chained predictors that compute the angle of the move, the magnitude of the move (e.g. learning rate) and the resolution of the grid of the zeroth-order samples at the landing point. We train our navigation agent on hard \emph{prototypical 2D surfaces} in order to make sure we develop feature detectors and subsequent policies that will be able to lead to good decisions in difficult areas of the loss function. We pose both the learning rate and resolution predictor as reinforcement learning problems and introduce a reward-shaping formula that allows us to learn from functions with different magnitude and from multiple proto-families. In our experiments this was a crucial factor in being able to generalize on many different types of evaluation functions.
   	
	We show that this setup leads to very good convergence speeds both in two and higher dimensions, on evaluation functions that are not presented at training time. For a zeroth-order optimization algorithm, the convergence performance is surprisingly good, leading to results comparative to or better than the task-specific optimizer (\emph{e.g} the best one out of set of specifically tuned first and second order optimizers).
	
	 \paragraph{} In conclusion, we believe that our main contributions are the following: framing the learning to optimize problem as a navigation task,  proposing a zeroth-order information-based learning architecture, coupled with a proper training procedure on prototypical two-dimensional surfaces and a reward shaping formula and showing experimentally that it can match/outperform first and second order techniques on meta-generalization tasks.
	
	
	\section{Organization of the report}
	This report is organized as follows. We first give a few preliminary notions of deep learning and reinforcement learning in Chapter \ref{chap::prel} that will show useful in this report. We give a brief summary of past and recent related work in the field of learning to learn and learning to optimize, and position our approach with respect to existing work in Chapter \ref{chap::related}. We then develop in \ref{sec::twod} our approach in the two-dimensional case, before extending it to a higher dimensional setting in \ref{sec::highd}. We present experimental results in Chapter \ref{chap::results} that show the validity of our approach in a variety of setups. We finally develop potential ideas for future work in Chapter \ref{chap::future}. 

	}
		
		\chapter{Preliminaries}
	{
		\label{chap::prel}
		\section{Deep learning}
		{
			\subsection{Definitions}
			\paragraph{} The aim of this section is not to give an exhaustive review of the very wide field of deep learning, but to refer the reader to useful pointers for models and techniques that would be used in this report. 
			
			\paragraph{} Deep-learning is not a recent idea (see \cite{wang2017origin} for a full historical review), but hardware development, advances in non-convex optimization and record-breaking performance in some applications (for instance in computer vision \cite{NIPS2012_4824}, speech recognition \cite{graves2013speech} and machine translation \cite{lecun1998gradient}) led it to become a largely adopted paradigm for machine learning practitioners as well as a wide and complex subject of research in machine learning. The key idea behind deep learning is to \emph{learn features} from data, by stacking layers of feed-forward predictors (see \cite{goodfellow2016} for complete and thorough details). More formally, a one hidden layer neural network produces an output $\mathbf{y}(\mathbf{x})\in\mathbb{R}^o$ given an input $\mathbf{x}\in\mathbb{R}^d$ such as:
			\begin{equation}
				\mathbf{y}(\mathbf{x}) = \sigma_o (V\sigma_h(W(\mathbf{x+b})+\mathbf{c})
				\label{eq::nnmodel}
			\end{equation}
			where $W\in\mathbb{R}^{h\times d}$ and $V\in\mathbb{R}^{o\times h}$ model the weights of the network, and $\mathbf{b}\in\mathbb{R}^d$, $\mathbf{c}\in\mathbb{R}^h$ its biases. $\sigma_h$ is the activation function for the hidden layer - typically a sigmoid, hyperbolic tangent or rectified linear (ReLu, \cite{nair2010rectified}) function. $\sigma_o$ is the output layer's activation, and is usually designed to fit task attributed to the neural model (linear for regression, softmax for multi-class classification, ..). As usual in machine learning, this model is then fed to a loss (statistically motivated or not) $\mathcal{L}(\mathbf{x},W,V,\mathbf{b},\mathbf{c})$ evaluating the quality of the results output by the network. 
			
			Reproducing this architecture over an over, one can increase the depth of the network - which is largely done in practice. It has been observed that deep networks have remarkable generalization properties, despite their high number of free parameters. It is indeed important to realize that the model's free parameters live in high dimension - $h\times d + o\times h + h + d$ only for our toy example \eqref{eq::nnmodel}. Furthermore, the loss $\mathcal{L}$ is non-linear and non-convex in these parameters, making it hard to optimize with standard optimization tools. 
			
			Since the first deep learning architecture , a great number of other architectures have been proposed (for instance \cite{he2016deep}, \cite{sabour2017dynamic}), and covering all of is out of the scope of this report. We'll focus here on the architectures that will be hereinafter used. The statistical motivation and numerical training of deep neural architecture will be evoked in \ref{chap::related}.
			
			\subsection{Convolutional layers}
			{
				\paragraph{} The idea of convolutional layers inherits from classical work in image and signal processing, where initially, hand-designed filters were used as feature detector in images. From a machine learning point of view, such filters can be learned directly from data if provided the right architecture. By enforcing \emph{parameter sharing} in the model, the weights of a convolutional layers belonging to the same filter are tied together, which leads to learning image features detectors. Convolutional layers can be stacked, and a flattened version of their output can be fed to classical, fully-connected (or sparse) layers. Tricks like pooling (dropping activations or applying a max-operator) are often applied to provide local rotation and translation invariance to the learned detectors. Figure \ref{fig::conv} provides a representation of the action of a convolutional layer. 
				
				\begin{figure}[h!]
					\begin{center}
						\includegraphics[width=0.8\textwidth]{typical_cnn}
						\caption{A typical convolutional architecture}
						\label{fig::conv}	
					\end{center}
				\end{figure}	
				
			}
			
			\subsection{Recurrent cells}
			{
				\paragraph{} A large number of machine learning problems involve sequence-to-label, or sequence-to-sequence predictions. If dealing with sequence of varying length, classical architectures fall short. A common approach to this problem is to use recurrent neural networks \cite{funahashi1993approximation}. Those networks maintain internal hidden states, that can get updated via recurrent connections, and therefore that can be seen as the hidden state of a dynamical system. Figure \ref{fig::rnn} display the vanilla architecture for a RNN. This naive approach have a major shortcoming, which is the vanishing our exploding nature of the gradient flow \cite{bengio1994learning} - linked to its  difficulties with long term dependencies. Various number of architectures have been proposed to cope with this problem, such as the Long-Short-Term-Memory (LSTM) cell \cite{hochreiter1997long} or the  Gated Recurrent Unit (GRU) \cite{chung2014empirical}.
				
				The training of such a model needs to be adapted from the training of classical feed-forward model, usually by \emph{unfolding the computational graph}. 
				
				\begin{figure}[h!]
					\begin{center}
						\includegraphics[width=0.8\textwidth]{typical_rnn}
						\caption{A typical recurrent architecture}
						\label{fig::rnn}	
					\end{center}
				\end{figure}	
			}
		}
		\section{Reinforcement learning}
		{
			\subsection{Formulation}
			{
							\subsubsection{Definitions}
			{
				\paragraph{} Reinforcement learning is a framework in which an \emph{agent} (or a \emph{learner}) learns its actions from interaction with its environment. The environment generates scalar values called \emph{rewards}, that the agent is seeking to maximize over time. 
			
				Let $\mathcal{S}$ denote the state space in which our agent evolves (the localization of a robot on a grid for instance), and $\forall{s}\in\mathcal{S}$ we will define the action state $\mathcal{A}(s)$, describing all possible action that can be taken by the agent at state $s$. When taking an action from a state $s_t$, the agent finds itself in a new state $s_{t+1}$ where it receives a reward $r_{t+1}\in\mathbb{R}$. The action taken is sampled over a probability distribution from the joint space of state and action: 
				\begin{equation}
					\begin{aligned}
						\pi \,  : \, \mathcal{S}\times\mathcal{A}(s) \, &\to [0,1]\\
							 (s,a) \, &\to \,  \pi(s,a)
					\end{aligned}
				\end{equation}	
				where $\pi(s,a)$ is the probability of picking action $a$ in state $s$. Such a distribution is called the agent's \emph{policy}. The key goal of reinforcement learning is teaching an agent on how to change its policy to maximize its reward on the long run. 
				The agent indeed seeks to maximize the \emph{expected return} $R_t$ mapping the reward sequence into $\mathbb{R}$. A commonly used expression for this value employs a \emph{discount factor} $\gamma \in [0,1]$, allowing to make the agent's more sensible to rewards it will get in a close future: 
				\begin{equation}
					R_t = \sum_{i=0}^T \gamma^i r_{t+1+i}
				\end{equation}
				This also allows to adapt this formulation to continuous tasks, where there are no terminal states and the task goes on indefinitely (there are no \emph{episodes} in the learning). 
			}
			\subsubsection{Markov Decision Process (MDP)}
			{
				\paragraph{} To make the problem tractable, we ask for the state signal to comply with Markov's property, hence to be \emph{memory-less}. For instance, we want to be able to write that, in a stochastic environment, $\forall s'\in\mathcal{S}$: 
				\begin{equation}
					\mathbb{P}\left( s_{t+1}=s' \, \vert \, a_t, s_t, \hdots, a_1,s_1\right) = \mathbb{P}\left( s_{t+1}=s' \, \vert \, s_t, a_t\right)
				\end{equation}
				
				Hence, every reinforcement learning problem can be represented by a \emph{Markov Decision Process}, that consists in a 5-tuple $\left(\mathcal{S}, \mathcal{A}, \mathcal{P}_{\cdot}(\cdot,\cdot), \mathcal{R}_{\cdot}(\cdot), \gamma \right)$ where: 
				\begin{itemize}[label=$\triangleright$]
					\item $\mathcal{S}$ is the agent's state space
					\item $\mathcal{A}$ is the agent's action space
					\item $\forall s,s'\in\mathcal{S}, \, \forall a\in\mathcal{A}(s)$,  $\mathcal{P}_a(s,s') = \mathbb{P}(s_{t+1}=s'\, \vert \, s_t = s, a_t = a)$ is the probability that action $a$ in state $s$ will lead the agent to transitioning to state $s'$.
					\item $\forall s,s'\in\mathcal{S}, \, \forall a\in\mathcal{A}(s)$,  $\mathcal{R}_a(s,s')$ is the immediate reward perceived by the agent when transitioning from state $s$ to $s'$ when taking action $a$. 
					\item $\gamma$ is the discount factor. 
				\end{itemize}
				
				A more realistic variant of MDP holds the name of \emph{Partially Observable Markov Decision Process} (POMDP), defined to be the tuple $\left(\mathcal{O},\mathcal{S},\mathcal{A},\mathcal{P},\mathcal{P}_o,\mathcal{R}\right)$ where $\mathcal{O}$ is the set of observations and $\mathcal{P}_o(o\vert s)$ the distribution of an observation conditionally to a state. The policy is now defined over observations $\pi(a\, \vert \, o)$ since the state is unaccessible to the agent (hence the name partially observable). 
			}
			\subsubsection{State and action value function}
			{
				\paragraph{} Most of the reinforcement learning algorithms are based on value function evaluation. A value function is a function mapping the state space in $\mathbb{R}$, estimating how good (in terms of expected future reward) it is for the agent to be in a given space. More precisely, a value function $V^\pi(\cdot)$ evaluates the expected return of a state when following the policy $\pi$. $V^\pi(\cdot)$ is called the \emph{state-value function}. 
				\begin{equation}
					\forall{s}\in\mathcal{S}, \quad V^\pi(s) = \E[\pi]{R_t \, \vert \, s_t = s}
				\end{equation}
				The \emph{action-value function} evaluates the value of taking a given action, and then following the policy $\pi$: 
				\begin{equation}
					\forall{s,a}\in\mathcal{S}\times\mathcal{A}(s), \quad Q^\pi(s,a) = \E[\pi]{ R_t \, \vert \, s_t=s, \, a_t=a}
				\end{equation}
				
				\paragraph{} Both those functions satisfy particular recursive relationships known as the \emph{Bellman equations}. It is shown that (see \cite{sutton1998reinforcement}) we have the following results: 
				\vspace{10pt}
				
				\coolbox{white}{\textcolor{black}{Bellman equations for Markov Decision Process}}
				{
					\begin{itemize}[label=$\triangleright$]
						\item Bellman equation for the state-value function: $\forall s \in\mathcal{S}$ 
						\begin{equation}
							V^\pi(s) = \sum_{a\in\mathcal{A}}\pi(s,a)\sum_{s'} \mathcal{P}_a(s,s')\left[\mathcal{R}_a(s,s') + \gamma V^\pi(s')\right]
						\end{equation}
						\item Bellman equation for the action value function: $\forall{s,a}\in\mathcal{S}\times\mathcal{A}(s)$: 
						\begin{equation}
							Q^\pi(s,a) = \sum_{s'}\mathcal{P}_a(s,s')\left[ \mathcal{R}_a(s,s') + \gamma V^\pi(s')\right]
						\end{equation}
					\end{itemize}
				}
			}
			\subsubsection{Optimal policies}
			{
				\paragraph{} The value functions define a partial ordering in the policy space. A policy $\pi$ is therefore said to be better than $\pi'$ (or $\pi\geq \pi'$) if $\forall{s}\in\mathcal{S}$, $V^\pi(s) \geq V^\pi(s')$. We are looking for $\pi^*$ so that: 
				\begin{equation}
					\forall\pi, \quad \pi^* \geq \pi 
				\end{equation}
				
				It was showed that for finite MDPs, there is always at least one policy that is better our equal to all others, and therefore is called the \emph{optimal policy} $\pi^*$. As shown in \cite{sutton1998reinforcement}, the state-value and action-value function verify the \emph{Bellman optimality equations}. 
				
				\vspace{10pt}
				
								\coolbox{white}{\textcolor{black}{Bellman optimality equations}}
				{
					\begin{itemize}[label=$\triangleright$]
						\item Bellman optimality equation for the state-value function: $\forall s \in\mathcal{S}$ 
						\begin{equation}
							V^\pi(s) = \max_{a\in\mathcal{A}(s)}\pi(s,a)\sum_{s'} \mathcal{P}_a(s,s')\left[\mathcal{R}_a(s,s') + \gamma V^\pi(s')\right]
						\end{equation}
						\item Bellman optimality equation for the action value function: $\forall{s,a}\in\mathcal{S}\times\mathcal{A}(s)$: 
						\begin{equation}
							Q^\pi(s,a) = \sum_{s'}\mathcal{P}_a(s,s')\left[ \mathcal{R}_a(s,s') + \max_{a\in\mathcal{A}(s')} Q(s',a') \right]
						\end{equation}
					\end{itemize}
				}
				\paragraph{} Those relations are essential in understanding the solving algorithms that will be presented later. 
			}
			}
			
			\paragraph{} There exists several ways of solving (i.e computing the optimal policy) a Markov Decision Process, that can generically be separated in two categories: \emph{model-based} and \emph{model-free} methods. 
			
			% TODO : I removed dynamic programing. Just say what it is, it is model based and therefore doesn't interest us here.
		
		\subsection{Temporal differences methods}
		{
			\paragraph{} Temporal difference methods can be seen as a combination of dynamic programing and another kind of learning called Monte Carlo methods, where the expected return are approximated via sampling. Like dynamic programming, TD methods are said to bootstrap (meaning that they build their estimators through already estimated values), but are \emph{model-free} methods and learn from experience. 
			
			\paragraph{} The justification, proof of convergences and literature and those models is pretty wide, hence we will not cover them in this report. However, a full description of those methods can be found in \cite{sutton1998reinforcement}. 
			
			\subsubsection{On-policy method: SARSA}
			{
				\paragraph{} The SARSA algorithm is an \emph{on-policy} control method, meaning that the algorithm updates the value function and improves the current policy it is following. At state $s_t$, it chooses an action $a_t$ from its policy and follows it. After observing the reward $r_{t+1}$ and the next state $s_{t+1}$, it again chooses an action $a_{t+1}$ using a soft policy and performs a one-step backup: 
				\begin{equation}
					Q(s_t,a_t) \leftarrow Q(s_t,a_t) + \alpha\left[r_{t+1} + \gamma Q(s_{t+1},a_{t+1}) - Q(s_t,a_t)\right]
				\end{equation}
				It therefore relies on a 5-tuple $(s_t,a_t,r_{t+1},s_{t+1},a_{t+1})$ to perform the udpate, giving it the State Action Reward State Action (SARSA) name. 
				
				The convergence properties of SARSA depend on the nature of the policy's dependency on Q. Indeed, SARSA converges with probability 1 to the optimal policy as long as all the sate and actions pairs are visited an infinite number of time, and the policy converges in the limit to the greedy policy. This is done, for instance, by turning the temperate of a softmax based policy to 0, or by having $\eps\to 0$ for a $\eps$-greedy policy. For SARSA to converge, we also as the learning rate to comply with the stochastic approximation conditions: 
					\begin{equation}
						\sum_k \alpha_k(a) = +\infty \quad { and } \quad \sum_k \alpha_k(a)^2 < +\infty
					\end{equation}
					where $\alpha_k(a)$ is the learning rate for the k\textsuperscript{th} visit of the pair $(s,a)$. 
					
			}
			\subsubsection{Off-policy method: Q-learning}
			{
				\paragraph{} The Q-learning algorithm is an off-policy method who learns to directly approximate $Q^*$, independently of the policy being followed. Its update rule is given by:
				\begin{equation}
					Q(s_t,a_t) \leftarrow Q(s_t,a_t) + \alpha \left[ r_{t+1} + \gamma \max_{a\in\mathcal{A}(s_{t+1})} Q(s_{t+1},a) - Q(s_t,a_t)\right]
				\end{equation}
				The actual policy being followed still has an effect in that it determines which state-actions pairs are visited and updated. However all that is required for convergence it that all pairs continue to be updated. 
				
				Along with this hypothesis and a slight variation in the usual stochastic approximation conditions, the learned action value function by Q-learning has been shown to converge to $Q^*$ with probability $1$. 
				
				In some cases, off-learning policies algorithms (like Q-learning) and on-policy ones (like SARSA) can learn different optimal policies (see \cite{sutton1998reinforcement}). This is mainly due to the fact that Q-learning performs update like if it was following a greedy policy - which it is not. That leads it to be less sensitive to a possible behavioral policy failure. 
			}
}
			
			\subsection{Policy search}
			{
				\label{sec::rlps}
				\subsubsection{Intuition}
				{
			\paragraph{} With continuous policy iteration, we approximated values functions:
			\begin{equation}
			\left\{
				\begin{aligned}
					&V_\theta(s) \simeq V^\pi(s) \\
					&Q_\theta(s,a) \simeq Q^\pi(s,a)
				\end{aligned}\right.
			\end{equation}
			and derived a policy (e.g $\eps$-greedy) from those approximations. With policy search, we now parametrize directly the policy:
			\begin{equation}
				\pi_\theta (s,a) = \mathbb{P}_{\pi_\theta}(a\vert s,\theta)
			\end{equation}
			
			\paragraph{} They are a few advantages to this approach, such as: better convergence properties, easier to tract in high-dimensional and continuous spaces, ability to learn stochastic policies (which is especially useful in POMDPs, where the optimal policy might be stochastic), $\hdots$ but also some disadvantages (presence of many local minima, high-variance and difficulty of policy evaluation,..)
				}
				\subsubsection{Objective functions}
				{
				\paragraph{} Given $\pi_\theta(s,a)$, what's the best $\theta$ ? We need a scalar evaluation of the parametrized policy. In episodic MPDs, we can use the start value as such a metric:
				\begin{equation}
					\begin{aligned}
						J_1(\theta) &= V^{\pi_\theta}(s_1)\\
								  &= \mathbb{E}_{\pi_\theta}(v_1)
					\end{aligned}
				\end{equation}
				
				In continuous MDPs, we can use the average value: 
				\begin{equation}
					J_{av}(\theta) = \sum_s d^{\pi_\theta}(s) V^{\pi_{\theta}}(s)
				\end{equation}
				with $d^{\pi_\theta}(s)$ is the expected number of time $t$ on which $s_t=s$ in a randomly generated episode - or equivalently a stationary distribution of the Markov chain induced by $\pi_\theta$ in the environment. 
				One can also use the \emph{average-reward} per time-step:
				\begin{equation}
					J_R(\theta) = \sum_s d^{\pi_\theta}(s) \sum_{a} \pi_\theta(s,a) \mathcal{R}_s^a
				\end{equation}
				}
				
				\subsubsection{Policy optimization}
				{
				\paragraph{} There are many ways to optimize those objective functions over $\theta$ - the more efficient being gradient-based methods. 
				Like any gradient-based method, our parameter update will take the form of: 
				\begin{equation}
					\Delta \theta \propto \grad[\theta]{J}(\theta)
				\end{equation}
				The real question here is how to properly evaluate $\grad[\theta]J(\theta)$. 
				
				
				We hereinafter derive an analytical solution for computing the policy gradient. We assume $\pi_\theta$ to be differentiable whenever it is non-zero, and that $\grad[\theta]\pi_\theta(s,a)$ is computable. We are also going to use the \emph{likelihood-ratio} trick: 
					\begin{equation}
						\grad[\theta]{\pi_\theta}(s,a) = \pi_\theta(s,a) \grad[\theta]{\log{\pi_\theta(s,a)}}
					\end{equation}
					If we consider one-steps MDPs, starting in state distributed along $d(s)$, terminating after receiving the one-step reward $r=\mathcal{R}_{s,a}$. Then: 
					\begin{equation}
						\begin{aligned}
							J(\theta) &= \E[\pi_\theta]{r}
								      &= \sum_{s\in\mathcal{S}} d(s) \sum_{a\in\mathcal{A}}  \pi_\theta(s,a) \mathcal{R}_{s,a} 
						\end{aligned}
					\end{equation}
					and then:
					\begin{equation}
						\begin{aligned}
							\grad[\theta]J(\theta) &= \sum_{s\in\mathcal{S}} d(s) \sum_{a\in\mathcal{A}} \pi_\theta(s,a) \grad[\theta]{\log{\pi_\theta(s,a)}}\mathcal{R}_{s,a}\\
											&= \E[\pi_\theta]{\grad[\theta]{\log{\pi_\theta(s,a)}}r}
						\end{aligned}
					\end{equation}
				
				The policy-gradient theorem \cite{sutton2000policy} allows this generalization by replacing $r$ with the long-term value $Q^{\pi_\theta}(s,a)$. The REINFORCE algorithm \cite{williams1992simple} is one kind of Monte-Carlo Policy-Gradient methods, in which we use the return $v_t$ as an unbiased sample of $Q^{\pi_\theta}(s_t,a_t)$. The update therefore writes:
					\begin{equation}
						\Delta\theta_t = \alpha \grad[\theta]\log{\pi_{\theta}(s_t,a_t)}v_t
					\end{equation}


				\paragraph{Actor-Critic Policy Gradient} Monte-Carlo Policy-Gradient has a very high variance because of the way we approximate the expected return. We might want to use a critic to estimate the action-value function:
					\begin{equation}
						Q_w(s,a) \simeq Q^{\pi_\theta}(s,a)
					\end{equation}
					We maintain two sets of parameters:
					\begin{equation}
						\left\{\begin{aligned}
						&\text{Critic: } \, \text{Action-value function parametrized by } w\\
						&\text{Actor: }\, \text{Policy parameter } \theta
						\end{aligned}\right.
					\end{equation}
					and we follow an approximate policy-gradient:
					\begin{equation}
						\Delta \theta = \alpha \grad[\theta]{\log\pi_\theta(s,a) Q_w(s,a)}
					\end{equation}
					The critic is solving this time the problem of policy evaluation. This can be done through MC policy evaluation, TD(0), TD($\lambda$). 


				\paragraph{Variance reduction} For any function $b(s)$, the expectation of the score function times $b$ is zero. Indeed:
					\begin{equation}
						\begin{aligned}
							\E[\theta]{\grad[\theta]{\log\pi_\theta(s,a)b(s)}} &= \sum_{s\in\mathcal{S}} d^\pi(s) \sum_{a\in\mathcal{A}} \grad[\theta]{\pi_\theta(s,a)b(s)} \\
							&= \sum_{s\in\mathcal{S}} d^\pi(s) b(s) \grad[\theta]{\sum_{a\in\mathcal{A}} \pi_\theta(s,a)}\\
							&= \sum_{s\in\mathcal{S}} d^\pi(s) b(s) \grad[\theta]{1}\\
							&= 0
						\end{aligned}
					\end{equation}
					We can therefore, without changing the expectation, reduce the variance by adding a baseline. We can henceforth derive, for every method, an optimal baseline. With actor-critic methods, a good baseline is the state-value function:
					\begin{equation}
						b(s) = V^{\pi_\theta}(s) 
					\end{equation}
					We can therefore rewrite the policy gradient using the advantage function $A^{\pi_\theta}(s,a) = Q^{\pi_\theta}(s,a) - V^{\pi_\theta}(s)$ and therefore:
					\begin{equation}
						\grad[\theta]{J(\theta)} = \E[\pi_\theta]{\grad[\theta]\log\pi_{\theta}(s,a) A^{\pi_\theta}(s,a)}
					\end{equation}
					This trick greatly reduces the variance of the policy gradient. Practically, we can use two function approximators and two parameter vectors: 
					\begin{equation}
						\left\{
						\begin{aligned}
							&V_v(s) \simeq V^{\pi_\theta}(s) \\
							&Q_w(s,a)  \simeq Q^{\pi_\theta}(s,a) \\
							&A(s,a) = Q_w(s,a) - V_v(s)
						\end{aligned}
						\right.
					\end{equation}
					Actually, for the true value function, the TD error $\delta^{\pi_\theta}$ is an unbiased estimator of the advantage function:
					\begin{equation}
						\begin{aligned}
							\E[\pi_\theta]{\delta^{\pi_\theta}\vert s,a} &= \E[\pi_\theta]{r + \gamma V^{\pi_\theta}(s') - V^{\pi_\theta}(s)\vert s,a}\\
							&= Q^{\pi_\theta}(s,a) - V^{\pi_\theta}(s) \\
							&= A^{\pi_\theta}(s,a)
						\end{aligned}
					\end{equation}
					Therefore it can be easily used directly (or through a proxy) to update the policy, which requires maintaining only one set of parameters. 
					
					}
				}
			}

					

	
	\chapter{Related work}
	{
		\label{chap::related}
		\section{Numerical optimization}
		{
			\paragraph{} The field of optimization has been studied for many years and for a great diversity of problems. Providing a complete review of the subject would be out of the scope of this paper, and therefore we provide only a short reminder on different approaches in the domain.  
	
	Some simple settings (convexity, $L$-smoothness, ..) have been intensively exploited to devise a large number of optimizers, derive upper-bound convergence rates (\cite{nemirovskii1983problem}) and even some information theoretical complexity lower bounds for black-box optimizers \cite{agarwal2009information}. More recently, motivated by the growing interest in deep learning, a lot of research efforts were also invested in devising smart, adaptative optimizers for complicated, very high dimensional objectives. Some of the most famous ones include AdaDelta \cite{zeiler2012adadelta}, RMSProp \cite{hinton2012rms} and Adam \cite{kingma2014adam}. Others authors focused on leveraging the particular structure of deep-learning architecture to efficiently compute, use curvature information and sometime overcome difficulties of curvature-based updates - \cite{vinyals2012krylov}, \cite{martens2010deep}, \cite{arjovsky2015saddle} are good examples of this trend. The general interest for deep learning also impulsed the rebirth of old methods that showed, after a few adaptations, promising results - \cite{hazan2016graduated}. 
	
	\paragraph{}Part of the diversity of existing optimizers is explained by the different type of oracles available for a given problem. Oracle-models are often considered in numerical optimization, for cases in which the literal expression of the function is unknown, or just too complicated to use. The optimization algorithm can only make queries to access a black-box like information $o_t$ of the function at time $t$. For a general function $f$ (possibly differentiable) mapping some multivariate parameter $\theta$ to $\mathbb{R}$, typical oracles involve zeroth-order oracle $o^0_t = \left[f(\theta_t)\right]$ and first order oracles $o^1_t = \left[f(\theta_t),\partial f(\theta_t)\right]$ with $\partial f$ denoting the sub-gradient operator of $f$. It is also common to consider noisy version of those oracles - for instance $\hat{o}_0^t = \left[f(\theta_t)+\xi\right]$ with $\xi$ a random variable, usually with finite first and second moments. The case of noisy first order oracles has been widely adopted in the machine learning community and led to many innovations (a detailed survey can be found in \cite{bottou2016optimization}). Noisy zeroth order oracles also received a lot of attention from the bandit community \cite{agarwal2011stochastic}, the Bayesian optimization community \cite{shahriari2016taking} and the evolutionary optimization communities (for which one of the most successful method being Covariance Matrix Adaptation Evolutionary Strategy \cite{hansen2016cma}). It have also seen a few heuristics approaches (the Nelder-Mead heuristic \cite{nelder1965simplex} being one of them). 
		}
		
		\section{Meta-learning}
		{
			\paragraph{} Learning to learn (or meta-learning) is not a recent idea. \cite{schmidhuber1987} thought of a Recurrent Neural Network (RNN) able to modify its own weights, building a fully differentiable system allowing the training to be learned by gradient descent. \cite{hochl2l} proposed to discover optimizers by gradient descent, optimizing RNNs modeling the optimization sequence with a learning signal emerging from backpropagation on a first network. 
	
	Recently, some meta-learning tentatives have shown great progress in different optimization fields. Various attempt tried to dynamically adapt the hyper-parameters of hand-designed algorithms, like \cite{daniel2016learning} or \cite{hansen2016using}. Using gradient statistics as an input for a recurrent neural net, \cite{li2017learning} were able to reinforcement learn a policy effective for training deep neural networks. In \cite{andrychowicz2016learning}, the authors show that when leveraging first-order information one could learn by gradient-descent optimizers that outperforms current state-of-the-art of existing problems - however only when the meta-train dataset is made of the same class of problem. When confronted to a different class of functions, the meta-learner is unable to infer efficient optimization moves. With the same idea of using gradient-descent for training the optimizer, \cite{chen2017learning} use zeroth order information in order to learn an optimizer for the Bayesian optimization setting.  

	\paragraph{} One could think that by showing enough examples to a meta-learner (namely made up of instances where traditional optimizers reach their limits), and adapting its structure to cover a large number of classes of functions, it could adapt to unknown loss landscapes. This idea was exploited by \cite{wil2l}, who manage to learn optimizers that generalizes to completely unseen data, while still being able to scale up to high-dimensional problems. Their process namely involves training by gradient descent hierarchical RNNs and showing it a great variety of examples. However, their optimizer's structure remains quite complicated, and doesn't provide human-level understanding of the features leveraged by the meta-learner. We believe that a more intelligible architecture could enable us to understand better what the network is learning, while still being effective on a large class of functions, when trained on a selected number of meta-examples. 
	
	\paragraph{}We also want to point out that if most of the method cited above call themselves meta-learning or learning to learn, they are closer to a \emph{learning to optimize} procedure. Indeed, none of their rewards or meta-loss are based on a validation or test loss, but on training loss. Shifting this paradigm tower learning to learn could reveal useful to investigate some current hot research topic on generalization in complex models - see \cite{keskar2016large} and \cite{dinh2017sharp}. However, to do so, the architecture and behavior of the meta learner have to be human-interpretable. Some recent works on meta-learning (\cite{ravi2016optimization} for instance) are closer to that problematic in the sense that they aim at learning good representations of \emph{what to learn} over whole datasets, that can be used later for \emph{few-shot learning}. 
	Recently, \cite{franceschi2017bridge} bridged the two domains of meta-learning and hyper-parameter optimisation. Therefore, work such as \cite{maclaurin2015gradient} can also be understood as meta-learning. This also opens a whole new exciting field of research, where the meta-learner is just seen as a complex model of hyper-parameters that needs to be optimized in some novel ways. 
		}
	}

	
	\chapter[Rover Descent]{Proposed approach: Rover Descent}
	{

\section{Intuition}
%	Hand-designed optimization algorithms, once well-tuned for a given problem, can already provide good baselines for providing optimization updates. However, they are susceptible to some degeneracies in the loss surface they are exploring (valley for gradient descent and many of its variants, saddle-point for curvature-based optimizer). Many efforts were recently put in overcoming these difficulties, or in trying to revise some older ideas in the optimization community to achieve efficient minimization of large-scale non-convex functions (a good example of this can be found in \cite{hazan2016graduated}). Deep-learning provides a good benchmarking environment for such problems and led to many innovations. We want to consider such strategies as teachers for our learning algorithm. As we noted before, it often happens that these well known optimizers each have a \emph{nemesis} landscape on which their improvement can be very slow or even inexistent. 
	
	In this paper, we propose framing the problem of learning to optimize as a problem of navigation on the partially observable error surface. The error surface is defined by the values of the loss function taken over the range of its inputs. In this framework, the optimizer is an agent that starting from the initial point, attempts to reach the lowest point on the surface with the smallest number of actions (where an action is a move to an arbitrary point on the landscape), while observing only a set of points sampled from the loss surface. Our goal is to learn the navigation policy that maps the current state of the agent to a move on the surface. 
	
	To this end, we decide to divide the architecture of our agent in three sequential modules: the normalized update direction predictor $\Delta$, that predicts the angle of the update, the learning rate predictor that predicts the magnitude of the update $\alpha$, and the resolution predictor, that predicts the scale $\delta$ of the observation set at the landing point. This choice is motivated by our intuition that these steps can be approached in a hierarchical way (first choose a direction, then a step size accordingly for instance) and therefore might involve different training methods and procedures. Furthermore, each of the modules can act as a correction factor on the other two modules. For example, if the update angle is not correct, the learning rate module can compensate by making the move very small and the resolution module can zoom out/in to make the next angle prediction task easier. Figure \ref{fig::archi} sums up the architecture we just devised in the two-dimensional case.
	
	%We decompose the update of our optimizer in three stages: choosing an appropriate normalized direction of update $\Delta$, choosing a step-size $\alpha$ to apply, and updating the resolution $\delta$  of the grid.
			
		%We now consider the naive setup where an optimization algorithm has access to several different optimizers, and has to decide which one to use at every step of the optimization procedure. If provided with a basic understanding of the landscape around the current iterate, it could dynamically switch between the different optimizers to pick the one it judges most adapted to the situation -  and therefore be robust to a wide variety of landscapes. For such a system to be efficient, and to avoid having to design and tune new basis optimizers for every new problem, it should also learn to mimic their moves - or directly the move of the optimizer it judges best given a representation of the local landscape. 
		%This reasoning leads us to think that a meta-learner, that is trained on a predefined set of \emph{prototypical loss landscapes} that appear frequently in common optimization problems in order to select an appropriate teacher and mimic its move, could show robustness on a variety of surfaces, and generalize to complex problems that it has not seen during training. Its input should be expressive enough to classify efficiently the surroundings of the current iterate and to be able to reproduce the behavior of diverse optimizers. 
	
\section{The two-dimensional case}
\label{sec::twod}
In the following subsection, we consider the simple case $d=2$ to develop our experimental set-up. A generalization for higher dimensions can be found in Section \ref{sec::highd}. 
		
		\begin{figure}[h!]
				\begin{center}
					\includegraphics[width=0.7\linewidth]{archi}
					\caption[Decomposition of the optimization step in three independent modules]{Decomposition of the optimization step in three independent modules: angle prediction, step-size prediction and resolution prediction.}
					\label{fig::archi}
				\end{center}
			\end{figure}	
			


\subsection{Prototypical landscapes}}
	
	\paragraph{Prototypical landscapes} Because our end goal is to be able to optimize complex loss landscapes, we are interested in selecting a small but sufficiently large set of prototypical landscapes as our meta-training set $\mathcal{F}_{train}$. More precisely, we decide to target surface degeneracies that are common when learning the weights of deep neural networks. These namely include \emph{valleys}, \emph{plateaus}, but also \emph{cliffs} (\cite{bengio1994learning}) and \emph{saddles} (\cite{dauphin2014identifying}). We also consider it useful to add \emph{quadratic bowls} to that list, to provide simpler and saner landscapes. Figure \ref{fig::sample_from_metatrain} provides a visualisation of each of these landscapes as generated by our meta-training algorithm, for which details can be found in Appendix \ref{sec::landscape_detail}. Interestingly enough, all of these landscapes were listed in \cite{schaul2013unit}, which provides a collection of unit tests for optimization. In the line of this work, we estimate that learning a optimizer over such landscapes can result in a robust algorithm. Also, because it is frequent in real world applications to only have access to noisy samples of the function we wish to optimize, our framework should therefore provide noisy versions of the landscapes described hereinbefore. 
     		
\subsection{Input design}
Let us consider two classical optimizers: gradient-descent and Newton descent (see \cite{nocedal2006numerical} for complete details). While gradient descend can escape non-strict saddle points but shows shattering behavior insides valleys (see Figure \ref{fig::nemesis}), second-order methods can leverage curvature to make quick progress inside valleys. However, saddles are attraction points for such methods. 
	
	To get the best of both worlds, we want our local descriptor to be able to represent both first and second order information. Finite difference provides an easy way to approximate them from zeroth order sample of a function. However, the precision of finite difference can  be severely impacted by noisy oracles, although this can be alleviated by a pre-filtering of the function samples (like low-pass filtering for removing white noise). 
		\begin{figure}[h!]
			\begin{center}
				\includegraphics[width=0.7\linewidth]{nemesis_landscapes}
				\caption[Nemesis landscapes]{\emph{Left}: gradient descent has a shattering behavior in narrow valleys. \emph{Right}: saddle points are attractors for Newton descent.}
				\label{fig::nemesis}
			\end{center}
		\end{figure}
			
		Let $f$ the loss landscape we are optimizing, mapping $\Omega_f \subset \mathbb{R}^d$ into $\mathbb{R}$, and $\theta \in\Omega_f$. A natural way to describe the surroundings of $\theta$ is to sample a grid centered on $\theta$. Given a budget of $n^2$ samples and a resolution $\delta$, we note $s_\delta^n(f,\theta)$ the resulting two-dimensional grid.
		\begin{equation}
				s_\delta^n(f,\theta) \triangleq \left(f \left( \begin{pmatrix} \theta_1 \\ \theta_2 \end{pmatrix} - \delta\cdot\begin{pmatrix} i - n/2 \\ j-n/2\end{pmatrix}\right)\right)_{i,j\in \{1,\hdots,n\}^2}
		\end{equation}
		
		This state representation has three advantages; it allows us to have a human-understandable input to our model, represent the surroundings of the current iterate and can approximate the inputs taken by gradient descent and Newton descend via finite difference. Another advantage is that pre-filtering can be efficiently applied by convolutions. For this state representation to represent compactly various functions, independent of their magnitude, we linearly rescale it to take its values in $[0,1]$.
			
		It is important to note that such an input becomes extremely expensive to compute as the dimensionality of the problem grows, as the size of $s_\delta^n(\cdot)$ grows exponentially with $d$. Therefore, we will use this solution for $d=2$, and discuss different ways of scaling to higher dimension in \ref{sec::highd}. 
			
		One could argue that using noisy zeroth order oracle for optimization is uncompetitive compared to higher order methods. Indeed, the study of convergence rates and lower bounds for convex optimization problem show the superiority of first-order oracles over single zeroth-order function evaluation (\cite{nemirovskii1983problem}). However, it was proven in \cite{duchi2015optimal} that by using two function evaluations, the oracle complexity of the latter type of algorithms could compete with the former, up to a low-order polynomial of the dimension factor (in the convex case). Because we use many of such samples, we are confident that we will be competitive against higher order oracles.
			
			\begin{figure}[h!]
				\begin{center}
					\includegraphics[width=0.9\linewidth]{ftrain_samples}
					\caption[Instances of $\mathcal{F}_{train}$]{Instances of $\mathcal{F}_{train}$. In order: quadratic bowl, valley, (plateau+cliff), saddle. Best viewed in color. }
					\label{fig::sample_from_metatrain}
				\end{center}
			\end{figure}
	
	%\vspace{-20pt}
	
	\subsection{Learning the update direction/angle}
	\label{sec::angle}
	The first step of our three-step optimizer is to determine a good direction of update, given a grid of samples $s_\delta^n(f,\theta)$. In light of the previous discussion, we decided to learn this angle prediction by \emph{imitation learning}, provided two teachers: gradient descent and Newton descent. The field of imitation learning is large, though dominated by two antagonist approaches: \emph{behavioral cloning} and \emph{inverse reinforcement learning}. The latter recovers the cost function that a teacher or expert is minimizing, while the former involves training a complex model (usually a deep neural network) in a supervised fashion so that it mimics a teacher. Thorough details on both these methods, as well as a complete survey of the field of imitation learning can be found in \cite{Billard2016}. Behavioral cloning, while being straight forward and simple to implement, is known to require a large amount of data and to be prone to \emph{compounding errors}, leading to divergence between the teacher's and the imitation followed paths. 
		On the other side, inverse reinforcement learning allows the imitator to interact with the environment, and fit its behavior over whole trajectories (therefore is not affected by the compounding error issue). However, it often implies using reinforcement learning in a inner loop, making this technique rather costly to use.  In our set-up, we decided to use a behavioral cloning approach. We can indeed easily generate large amounts of training data, and are not trying to fit the entire teacher behavior but only a subpart - the direction, not the step-size.
			
	We collect our training data by launching optimization runs, where we follow the best out of the teachers (here best means leading to the largest decrease of the objective function). At each step, we record the local grid sample with a pre-determined resolution, as well as two opposite directions of update: the optimal one $d_*$ (given by the best teacher) and a set of opposite randomly generated ones $\tilde{d_*}$ (sampled to lie in the half-space defined by $\{ d\, \vert \, d^Td_* <0\}$). The expert move $d_*$ and its negative counterparts $\tilde{d}_*$ are normalized to create two actions $a_*= d_* / \lVert d_*\rVert_2$ and $\tilde{a}_*$ similarly. We then create two state-action pairs with respective label $t=1$ and $t=0$, corresponding to a positive and a negative sample. In practice, we sampled $5\cdot 10^4$ functions from $\mathcal{F}_{train}$ and let the optimization procedure run for $10$ steps on each functions, creating $10^6$ (state,action,label) tuples to train on, stored in $\mathcal{D}_{train}$.
			
	To fit the resulting (state-action) pairs, we design a simple neural network made of two convolutional layers followed by two fully connected layers. The idea of using convolutional layer is related to the problem of filtering we mentioned earlier, and to the idea that the learnt filters in the convolutional layer can act as identifiers for the different landscapes encountered during an optimization run. For a given grid of sample $s$, we denote $y(\omega,s)$ the output of this model, parametrized by the weights $\omega$ of the network. We use batch-normalization layers after the convolutional layers, and train the model to minimize the cross-entropy loss:
			\begin{equation}
				J(\mathcal{D}_{train},\omega) = \sum_{(s,a,t)\in\mathcal{D}_{train}} \Big[t\log{\{\sigma(y(\omega ,s),a)\}} + (1-t)\log{\{1-\sigma(y(\omega,s),a)\}}\Big]
			\end{equation}
			with $\sigma(y(s),a) = (1+e^{-y(\omega,s)^Ta})^{-1}$.
			The objective is to learn to correlate the output of the model when the action $a$ has a positive label (it was sampled from the best teacher). The idea of storing negative versions of that optimal action can be understood as negative sampling, or noise contrasted estimation \cite{gutmann2010noise}. We found that this approach, over the other ones we tried, lead to better performances while greatly reducing overfitting. Because we only want to use this model as an angle predictor, we will use a normalized version of the output: $\Delta(s) = y(s) / \rVert y(s) \lVert_2$.
	
		\subsection{Learning the step-size and the resolution}
		\label{sec::rl_reso}
		
		 At this point, we have learnt a good angle predictor. We now want to learn two new behaviors: the step-size to apply to the update, as well as the next resolution of the sample grid. Learning the step-size is obviously crucial for the optimization step. Learning the resolution is also extremely important: far from an optimum, we'd like to zoom-out to get a better understanding of the landscape. Close to an optimum, we expect an efficient system to zoom in to refine its estimation of the localization of the optimal point. Those two behaviors can't be learnt efficiently from a teacher (line-search is an unfairly good teacher for the step-size, and we simply don't have available a hand-designed teacher for the resolution).
			
		\paragraph{Deterministic Policy Search} Policy search is a family of algorithm that directly search in the policy space for $\pi^* = \argmax{\pi}{\bar{R}}$ - see \ref{sec::rlps}. To make this search tractable, $\pi$ is usually tied to some parametrized family. A popular algorithm to perform that search is the Deterministic Policy Gradient \cite{silver2014deterministic} (DPG) where we learn a deterministic parametrized policy $\pi_\eta(a\vert s) = \mu(\eta,s)$ in a fully observable Markov decision process ($\mathcal{O}=\mathcal{S}$). The system is composed of two entities, an actor and a critic. The critic, parametrized by $\omega$, has the role to evaluate the Q-values (expected return when taking an action $a$ in state $s$) of the current policy induced by the actor (parametrized by $\eta$). As it is common in actor-critic approaches, the critic is updated by batch of logged experience to minimize the squared temporal difference (TD) error $\left(r_t + \gamma Q_\omega(s_{t+1},a_{t+1}) - Q_\omega(s_t,a_t)\right)^2$. The actor's parameters are updated in the direction that maximizes the Q-values for a batch of logged states: $\Delta \eta \propto \nabla_a Q_\omega(s,a)^T \nabla_\eta \mu(\eta,s)$. \cite{lillicrap2015continuous} applied this algorithm to deep neural networks as function approximators, using techniques that were proven successful in deep Q-learning \cite{mnih2015human}, like target networks and experience replay. \cite{heess2015memory} also extended this approach for POMDP, where it is useful to use Recurrent Neural Networks as models for the policy.
			
		\paragraph{Reinforcement learning formulation} We consider the following environment for our problem. Let, for a given loss function $f$, the full state space $\mathcal{S}_f = \left\{ \theta, \alpha, \delta \right\}$ and the observations $\mathcal{O}_f = \left( s_\delta^n(f,x) \right)$. The agent hence only has access to the current grid of samples around $\theta$ with resolution $\delta$, but not to the current iterate position $\theta$, the current-step size $\alpha$ or the current resolution $\delta$. The idea behind this is to be able to generalize to unseen landscapes, and be robust to transformations such as rescaling or translations. The only events that should impact the agent's behavior is a sharp change in the neighboring landscape around the current iterate. The action space is set to be $\mathcal{A} = \{ \Delta \alpha , \Delta \delta \}\subseteq [-0.5,1]^2$ which constitutes the update rate of the step-size and the resolution. We consider deterministic transitions:
			\begin{equation}
				\begin{aligned}
					&\theta_{t+1} = \theta_{t} + \alpha_t \Delta(s_{\delta_t}^n(f,\theta_t)) \\
					&\alpha_{t+1} = \alpha_t(1+\Delta \alpha_t) \\
					&\delta_{t+1} = \delta_t(1+\Delta\delta_t)
				\end{aligned}
				\label{eq::transitions}
			\end{equation}
			where the current iterate $\theta_t$ is updated along the direction $\Delta(s_{\delta_t}^n(f,\theta_t))$ with step-size $\alpha_t$. 
			
			We have several options for the reward function. One possibility is be to consider a budgeted optimization scheme, with reward $r(s) = -f(\theta_t) \mathds{1}_{t=T}$ (the reward is only given by the final value of the function at the last step). In this case, the reward is rather sparse, and leads the trajectory search in ambiguous ways. We can prefer another solution, where the whole trajectory of the agent over the landscapes is evaluated: $ r_f(s_t) = -f(\theta_t)$. 
			This leads the policy search to optimize for the following return:
			\begin{equation}
				R_f = -\sum_{t=0}^T \gamma^t f(\theta_t)
				\label{eq::reward}
			\end{equation}	
			Note that for $\gamma = 1$ this leads us to optimize over the same criterion that \cite{li2017learning} and \cite{andrychowicz2016learning} (in the former, the authors call this the meta-loss). 
			
		It is important to note that the previously described POMDP, that we will denote as $\mathcal{M}_f$, is parametrized by a function $f$ sampled inside $\mathcal{F}_{train}$. This induces a distribution $\mathcal{M}_{train}$ over POMDPs. In the following experiments, we won't make that distinction and train a single parametrized policy on the resulting POMDP distribution - that implies that every new episode is generated with $\mathcal{M}_f \sim \mathcal{M}_{train}$. This induces a difficulty over the learning task: both the transitions and the reward defined in \eqref{eq::transitions} and \eqref{eq::reward} change between every episode. To help the agent figure out optimal moves, we can change the reward so that it becomes insensitive to the magnitude of the sampled function $f$ and the position of the initial iterate $\theta_0$:
			\begin{equation}
				r_f(s_t) = -\frac{f(\theta_t) - f(\theta_f^*)}{f(\theta_0)-f(\theta_f^*)}
			\end{equation}
			with $\theta_f^* = \argmin{\theta}{f(\theta)}$. To also help the agent optimize over long trajectories where the magnitude of $f(\theta_0)$ largely surpasses $f(\theta_f^*)$, we propose a second version of the reward function:
			\begin{equation}
				r_f(s_t) = -\frac{f(\theta_t) - f(\theta_f^*)}{\bar{f}_k-f(\theta_f^*)}
			\end{equation}
			with $\bar{f}_k$ being the mean value of the objective function over the last $k$ iterates (we found that in our set-up, $k=5$ provides good results). The use of this reward function was a crucial element in the success of our reinforcement learning approach.
			
	We model the agent policy by a recurrent neural network, made up of two convolutional layers, followed by a LSTM cell, followed itself by two hidden layers. The critic is modeled by a similar network, and both were trained using the DPG algorithm. During training, we sample $f\sim\mathcal{F}_{train}$ at the beginning of each episode. The initial iterate is randomly sampled in the landscape so that it is far away enough from the optimum of the loss function. The episode is ran for a fixed horizon $T=30$ and we fix the discount factor $\gamma$ to 1. 
	
\section{Architecture for $d>2$}
\label{sec::highd}

	The idea of using grid samples $s_\delta^n(f,\theta)$ can't be exploited in high-dimensional problems as its size grows exponentially with $d$. To extend our framework for $d>2$, we consider the following set-up: let $f: \mathbb{R}^d \to \mathbb{R}$ and $\theta$ the initial iterate. We note $s(f,\theta,i,j)$ the vector that contains the two-dimensional grid sampled at $\theta$ along the dimensions $i$ and $j$. In other words, with $\delta_i$ the $d$-dimensional vector whose entries are all 0 but the ith one that is set to 1, and $E_{i,j} = (\delta_i,\delta_j)$ a $d\times2$ matrix, we note: $s_\delta^n(f,\theta,i,j) = s_\delta^n(f,E_{i,j}^T\theta)$.
		
		By considering all pairs of dimensions, we can compute $d(d-1)/2$ of such grids, leading to the prediction of as many angles $\Delta_{i,j}(f,\theta) = \Delta(s_\delta^n(f,\theta,i,j))$, step-size updates $\Delta \alpha_{i,j}$ and resolution updates $\Delta \delta_{i,j}$ - all predicted with the models trained in the two dimensional case. Therefore, if we keep record of all step-size and resolution for every pair of dimension $(i,j)$ we can compute $d(d-1)/2$ updates $\Delta \theta_{i,j} = \alpha_{i,j} \Delta_{i,j}(f,\theta)$. We can consider each of these outputs like the $d(d-1)/2$ two-dimensional projections of the true $d$-dimensional update $\Delta \theta$ so that $\Delta \theta_{i,j}= E_{i,j}^T \Delta \theta$. We can therefore try to retrieve $\Delta\theta$ by a least-square approach, and find $\Delta \hat{\theta}$:
		
		\begin{equation}
			\Delta \hat{\theta}\triangleq \argmin{\delta \theta}{ \sum_{1\leq i < j \leq d} \left(E_{i,j}^T \delta\theta - \Delta\theta_{i,j}\right)^2}
		\end{equation}
		
		Solving this equation leads to the analytical expression:
		\begin{equation}
			\begin{aligned}
				\Delta \hat{\theta}
							&= \frac{1}{d-1} \sum_{1\leq i < j \leq d} \alpha_{i,j} E_{i,j}\Delta_{i,j}(f,\theta)
							%&= \frac{1}{d-1} \sum_{1\leq i < j \leq d}E_{i,j}\Delta\theta_{i,j}\\
			\end{aligned}
		\end{equation}
		
		Each pair of dimension has a corresponding step-size $\alpha_{i,j}$  and resolution $\delta_{i,j}$, which are updated by running the associated two-dimensional grid through the system described in \ref{sec::rl_reso}. This computation requires maintaining $d(d-1)/2$ learning rates and resolutions, and computing as many grid samples. Because of this quadratical growth with the dimension, this leads to clock-time and memory issues for large values of $d$. A simple way round this problem is to sample $k<d$ pairs of dimensions, compute $\Delta\hat{\theta}$ based only on these $k$ pairs and update their corresponding learning rates and resolutions. If we note $\Theta_k$ the set of $k$ pairs we sampled:
		
		\begin{equation}
			\Delta \hat{\theta}_k =  \frac{1}{k-1} \sum_{(i,j)\in\Theta_k} \alpha_{i,j} E_{i,j}\Delta_{i,j}(f,\theta)
			\label{eq::d_update}
		\end{equation}
		Different strategies can be employed to sample $\Theta_k$, possibly leveraging some knowledge about the optimization problem's structure. Such strategies are experimentally investigated in \ref{sec::net_exp}.
				
	}
	
	%\vspace{-10pt}

\chapter{Experimental results}
	\label{chap::results}
	\section{Behavioral analysis}
	{
		
		\paragraph{Angle predictor} The training of the angle predictor is straight forward and leads to robust angle prediction. Table \ref{tab::angle_result} shows the mean angle dissimilarity between the learnt angle predictor and the best teacher (as defined in \ref{sec::angle}) on a set of held-out functions from $\mathcal{F}_{train}$. We show this results when training and testing on either single modalities (e.g only quadratic, only valleys, ..) of $\mathcal{F}_{train}$, or all of them at the same time. We see that training the angle predictor on the whole meta-dataset does not impact its predictions abilities compared to landscape-specific training. Indeed, we only see a small drop in the quality of the predictions, which we attribute to the partial observability of the local landscapes (a valley seen under a small resolution can locally appear like a quadratic bowl, for instance). 
			\begin{table}[h!]
				\centering
				\begin{tabular}{|c|c|c|c|c|}
				\hline
			 	\backslashbox{Training}{Testing} & Quadratics & Valleys & Saddles & Plateaus+cliffs\\
				\hline
				Quadratics & 1.2 &  89.2 & 43.1 & 34.8\\
				Valleys & 83.5 & 4.9 & 94.1 & 86.9 \\
				Saddles & 44.7 & 85.8 & 3.8 & 57.2 \\
				Plateaus+cliffs & 29.4 & 94.3 & 53.9 & 1.9 \\
				\hline 
				\hline 
				All & 3.1 & 6.4 & 5.1 & 3.8 \\
				\hline
				\end{tabular}
				\caption{Mean angle dissimilarity in degrees on held-out functions from $\mathcal{F}_{train}$ when training on both single modalities and the whole meta-dataset.}
				\label{tab::angle_result}
			\end{table}
		
		\paragraph{Resolution and step-size predictor} Evaluating exhaustively the learnt policy for updating the resolution and the step-size is complicated. In Figure \ref{fig::policy_seq}, we display the dynamics of the step-size and the resolution chosen by the learnt policy, on instances of $\mathcal{F}_{train}$. For each figure, a landscape is randomly drawn for $\mathcal{F}_{train}$ and the initial state (i.e initial step-size, resolution and initial iterate) is sampled in the same distribution used at training time. Independently of the quality of this policy (which will be evaluated in the following sections), we can notice it follows our intuition of what is a good policy in this context: it zooms out  and increases the step-size in the beginning of the iteration procedure, while zooming in and reducing the step-size near the end when the local landscapes indicates the presence of a minimum nearby. 
		
		\begin{figure}[h!]
			\centering
			\begin{subfigure}[b]{0.45\linewidth}
			{
				\centering
				\includegraphics[width=0.8\textwidth]{policy_seq_quad}
				\caption{Quadratic}
				\label{fig::policy_seq_quadratic}
			}
			\end{subfigure}
			\begin{subfigure}[b]{0.45\linewidth}
			{
				\centering
				\includegraphics[width=0.8\textwidth]{policy_seq_valleys}
				\caption{Valley}
				\label{fig::policy_seq_valley}
			}
			\end{subfigure}\\
			\begin{subfigure}[b]{0.45\linewidth}
			{
				\centering
				\includegraphics[width=0.8\textwidth]{policy_seq_saddle}
				\caption{Saddle}
				\label{fig::policy_seq_saddle}
			}
			\end{subfigure}
			\begin{subfigure}[b]{0.45\linewidth}
			{
				\centering
				\includegraphics[width=0.8\textwidth]{policy_seq_cliff}
				\caption{Plateau+cliff}
				\label{fig::policy_seq_cliff}
			}
			\end{subfigure}
			\caption{Step-size and resolution trajectories for different instances of $\mathcal{F}_{train}$.}
			\label{fig::policy_seq}
		\end{figure}
		
		
		This curves could suggest that the policy has only learnt good scheduling schemes for each of the modalities in $\mathcal{F}_{train}$. We show here that this is not the case, by showing that the policy is able to adapt to sharp changes of its initial state. In Figure \ref{fig::policy_seq_ad}, we sample this initial state far from the distribution that was used at training time (typically, a learning rate that is 100 times smaller that it usually is at training time, and a resolution 100 bigger), for a quadratic bowl. For reference, Figure \ref{fig::policy_seq_quad_ad} display the trajectory followed by the policy on the same landscape with a well set initial state. We can see that the schedule adapts to these changes, and therefore that mostly changes in the local landscape seen by the agent affects its decision (and not a fixed schedule learnt by the LSTM). 
		
		
		\begin{figure}[h!]
		\centering
			\begin{subfigure}[b]{0.45\linewidth}
			{
				\centering
				\includegraphics[width=0.8\linewidth]{policy_seq_quad_ad}
				\caption{Step-size and resolution trajectory with well set initial state.}
				\label{fig::policy_seq_ad}
			}
			\end{subfigure}
			\begin{subfigure}[b]{0.45\linewidth}
			{
				\centering
				\includegraphics[width=0.8\linewidth]{policy_seq_ad}
				\caption{Step-size and resolution trajectory with poorly set initial state.}	
				\label{fig::policy_seq_quad_ad}
			}
			\end{subfigure}
			\caption[Some policy trajectories]{Comparing policy's trajectories on a quadratic bowl with different settings of the initial state.}
		\end{figure}
	}
	\section{Two-dimensional experiments}
	{
		\subsection{Testing on $\mathcal{F}_{train}$}
		We first want to evaluate the optimizer resulting from the model we introduced on $\mathcal{F}_{train}$ to evaluate its behavior on known landscapes. This already can be seen as some kind of meta-testing on some hold-out since we only sample in $\mathcal{F}_{train}$, which contain an infinite number of functions. Therefore, we can assume that whenever we sample in $\mathcal{F}_{train}$, we will obtain a function that the optimizer has not seen during training. 
		
%		\begin{figure}[h!]
%			\begin{center}
%				\includegraphics[width=0.8\linewidth]{full_test_Ftrain}
%				\caption{Tests runs on instances of modalities of $\mathcal{F}_{train}$. From top left to bottom right: quadratic bowl, valley, (plateau+cliff),  saddle. The shades represents the envelope of the trajectories over an entire 20-fold. Best viewed in color.}
%				\label{fig::ftrain_test}
%			\end{center}
%		\end{figure}
		
		\begin{figure}[h!]
			\centering
			\begin{subfigure}[b]{0.45\linewidth}
			{
				\centering
				\includegraphics[width=0.9\textwidth]{best_quad}
				\caption{Quadratic}
				\label{fig::fquad_eval}
			}
			\end{subfigure}\hfill
			\begin{subfigure}[b]{0.45\linewidth}
			{
				\centering
				\includegraphics[width=0.9\textwidth]{best_valley}
				\caption{Valley}
				\label{fig::valley_eval}
			}
			\end{subfigure}\\
			\begin{subfigure}[b]{0.45\linewidth}
			{
				\centering
				\includegraphics[width=0.9\textwidth]{best_uni}
				\caption{Plateau+cliff}
				\label{fig::uni_eval}
			}
			\end{subfigure}\hfill
			\begin{subfigure}[b]{0.45\linewidth}
			{
				\centering
				\includegraphics[width=0.9\textwidth]{best_saddle}
				\caption{Saddle}
				\label{fig::saddle}
			}
			\end{subfigure}
			\caption[Tests runs on instances of modalities of $\mathcal{F}_{train}$]{Tests runs on instances of modalities of $\mathcal{F}_{train}$. The shades represents the envelope of the trajectories over an entire 20-fold. Best viewed in color.}
			\label{fig::ftrain_test}
		\end{figure}
		
		We follow a simple procedure: we sample $f\in\mathcal{F}_{train}$, an initial point $\theta_0 \in \Omega_0$, and sample uniformly at random an initial step-size and an initial resolution inside the distribution used at meta-training time. We then add a small perturbation to the initial iterate and run an optimization trajectory with a fixed horizon. We repeat this procedure many times to evaluate the global sensitivity of our algorithm to the position of the first iterate. To compare its performance with a broad variety of optimizers, we decide to evaluate with the same procedure a collection of optimization algorithms that include: gradient descent, Nesterov accelerated gradient descent, Newton Descent, Covariance Matrix Adaptation Evolution Strategy (CMA-ES) and the Nelder-Mead method. The results are regrouped in Figure \ref{fig::ftrain_test}. The lines represent the mean trajectory of each optimizer, while the shaded areas represent the envelope of all its trajectories (that were generated from noisy versions of the initial iterate).  The results, shown here for a single function $f$ and iterate point $\theta_0$ are consistent in our experiments - that we have learnt to compete with a wide variety of hand-designed algorithms, and that our optimizer is fairly independent of the initial value set for $\alpha_0$ and $\delta_0$. The hyper-parameters of hand-designed methods are modified at each time to perform as well as possible on the whole modality of $\mathcal{F}_{train}$ we are testing on. This means that our learnt optimizer sometimes compete with unfairly good algorithms (like Newton descent on a quadratic loss, that hits the optimum after just one iteration). In some cases, the apparent lack of trajectory envelope is due to the fact that the perturbation on the initial point sometimes have to be reduced for visualisation purposes. We provide in Appendix \ref{sec::policy_viz} visualizations of the trajectories followed by our learnt optimizer. 
		
		\subsection{Testing on $\mathcal{F}_{test}$}
		To evaluate the meta-generalization abilities of our learnt optimizer, we also evaluate it on a two-dimensional meta-testing dataset $\mathcal{F}_{test}$. We selected various two-dimensional optimization problems known to be challenging for general optimization methods. The complete list contains Rosenbrock, Ackley, Rastrigin, Maccornick, Beale and Styblinksi's function, for which the literal expressions and surface plots can be found in Appendix \ref{sec::ftest}. It is important to note that none of these landscapes were seen by the optimizer during its training. 
		
		For each of those functions, we select a starting point that constitute a challenge for all compared optimizers (also indicated on the surface plot in Appendix \ref{sec::ftest}). We then followed the previously described procedure, and set the hand-designed optimizers hyper-parameters to show good behavior for every small perturbation of the initial iterate. The results are displayed in Figure \ref{fig::styblinskis}, \ref{fig::beales}, \ref{fig::rosenbrocks}, \ref{fig::maccornicks}, \ref{fig::ackleys} and \ref{fig::rastrigins}, and remain consistent when changing the starting point for each of the meta-test function. Our learnt optimizer can generalize to new landscapes, even multimodal ones, and compete with a wide variety of optimizers. On these multimodal landscapes, CMA-ES and the Nelder-Mead method provide two strong baselines (only the mean trajectory appear for such functions for vizualisation purposes). On \ref{fig::ackley_eval}, they are the only two algorithms with our method that find the global optimum. However, in \ref{fig::rastrigin_eval}, only our method finds for every perturbation of the initial iterate the global minimum. Our optimizer starts by zooming out to get a better understanding of the landscape, leading it to quickly cover an area where lays the global minimum. We also provide single trajectories visualisation on contour plots (the blue square denoting the starting point, and the dark circle the position of the optimum).

%		\begin{figure}[h!]
%			\begin{center}
%				\includegraphics[width=0.8\linewidth]{full_test_Ftest}
%				\caption{Tests runs on instances of modalities of $\mathcal{F}_{test}$. From top left to bottom right: Styblinksi, Beale, Beale, Rosenbrock, Maccornick, Ackley, Rastrigin. The shades represents the envelope of the trajectories over an entire 20-fold. Best viewed in color.}
%				\label{fig::ftest_test}
%			\end{center}
%		\end{figure}

		\begin{figure}[h!]
			\centering
			\begin{subfigure}[b]{0.45\linewidth}
			{
				\centering
				\includegraphics[height=0.7\textwidth]{best_styblinksi}
				\caption{Function value trajectories}
				\label{fig::styblinksi_eval}
			}
			\end{subfigure}
			\hfill
			\begin{subfigure}[b]{0.45\linewidth}
			{
				\centering
				\includegraphics[height=0.7\textwidth]{fplot_styblinski}
				\caption{Contour plot trajectory}
				\label{fig::styblinksi_eval_traj}
			}
			\end{subfigure}
		\caption[Tests runs on Styblinski's function]{Tests runs on Styblinski's function. \emph{(Left)} The shades represents the envelope of the trajectories over an entire 20-fold. \emph{(Right)} The blue square denotes the initial point and the black circle the minimum.}
		\label{fig::styblinskis}
		\end{figure}
		
			\begin{figure}[h!]
			\centering
			\begin{subfigure}[b]{0.45\linewidth}
			{
				\centering
				\includegraphics[height=0.7\textwidth]{best_beale}
				\caption{Function value trajectories}
				\label{fig::beale_eval}
			}
			\end{subfigure}
			\hfill
			\begin{subfigure}[b]{0.45\linewidth}
			{
				\centering
				\includegraphics[height=0.7\textwidth]{fplot_beale}
				\caption{Contour plot trajectory}
				\label{fig::beale_eval_traj}
			}
			\end{subfigure}
		\caption[Tests runs on Beale's function]{Tests runs on Beale's function. \emph{(Left)} The shades represents the envelope of the trajectories over an entire 20-fold. \emph{(Right)} The blue square denotes the initial point and the black circle the minimum.}
		\label{fig::beales}
		\end{figure}
		
		\begin{figure}[h!]
			\centering
			\begin{subfigure}[b]{0.45\linewidth}
			{
				\centering
				\includegraphics[height=0.7\textwidth]{best_rosenbrock}
				\caption{Function value trajectories}
				\label{fig::rosenbrock_eval}
			}
			\end{subfigure}
			\hfill
			\begin{subfigure}[b]{0.45\linewidth}
			{
				\centering
				\includegraphics[height=0.7\textwidth]{fplot_rosenbrock}
				\caption{Contour plot trajectory}
				\label{fig::rosenbrock_eval_traj}
			}
			\end{subfigure}
		\caption[Tests runs on Rosenbrock's function]{Tests run on Rosenbrock's function. \emph{(Left)} The shades represents the envelope of the trajectories over an entire 20-fold. \emph{(Right)} The blue square denotes the initial point and the black circle the minimum.}
		\label{fig::rosenbrocks}
		\end{figure}
		
		\begin{figure}[h!]
			\centering
			\begin{subfigure}[b]{0.45\linewidth}
			{
				\centering
				\includegraphics[height=0.7\textwidth]{best_maccornick}
				\caption{Function value trajectories}
				\label{fig::maccornick_eval}
			}
			\end{subfigure}
			\hfill
			\begin{subfigure}[b]{0.45\linewidth}
			{
				\centering
				\includegraphics[height=0.7\textwidth]{fplot_maccornick}
				\caption{Contour plot trajectory}
				\label{fig::maccornick_eval_traj}
			}
			\end{subfigure}
		\caption[Tests runs on Maccornick's]{Tests runs on Maccornick's function. \emph{(Left)} The shades represents the envelope of the trajectories over an entire 20-fold. \emph{(Right)} The blue square denotes the initial point and the black circle the minimum.}
		\label{fig::maccornicks}
		\end{figure}
		
		\begin{figure}[h!]
			\centering
			\begin{subfigure}[b]{0.45\linewidth}
			{
				\centering
				\includegraphics[height=0.7\textwidth]{best_ackley}
				\caption{Function value trajectories}
				\label{fig::ackley_eval}
			}
			\end{subfigure}
			\hfill
			\begin{subfigure}[b]{0.45\linewidth}
			{
				\centering
				\includegraphics[height=0.7\textwidth]{fplot_ackley}
				\caption{Contour plot trajectory}
				\label{fig::ackley_eval_traj}
			}
			\end{subfigure}
		\caption[Tests runs on Ackley's function]{Tests runs on Ackley's function. \emph{(Left)} The shades represents the envelope of the trajectories over an entire 20-fold. \emph{(Right)} The blue square denotes the initial point and the black circle the minimum.}
		\label{fig::ackleys}
		\end{figure}
		
		\begin{figure}[h!]
			\centering
			\begin{subfigure}[b]{0.45\linewidth}
			{
				\centering
				\includegraphics[height=0.7\textwidth]{best_rastrigin}
				\caption{Function value trajectories}
				\label{fig::rastrigin_eval}
			}
			\end{subfigure}
			\hfill
			\begin{subfigure}[b]{0.45\linewidth}
			{
				\centering
				\includegraphics[height=0.7\textwidth]{fplot_rastrigin}
				\caption{Contour plot trajectory}
				\label{fig::rastrigin_eval_traj}
			}
			\end{subfigure}
		\caption[Tests runs on Rastrigin's function]{Tests runs on Rastrigin's function. \emph{(Left)} The shades represents the envelope of the trajectories over an entire 20-fold. \emph{(Right)} The blue square denotes the initial point and the black circle the minimum.}
		\label{fig::rastrigins}
		\end{figure}
		
	}
	
	%\vspace{-10pt}
	
	\section{High-dimensional experiments}
	{
		We now test the procedure described in Section \ref{sec::highd} for problem of dimensions $d>2$. We propose to do this by considering a linear classifier for a binary classification task, and a small neural network classifier. 
		
		\subsection{Linear binary classification}
		{
			We generate random binary classification tasks in dimension $d>2$, according to the framework described in \cite{guyon2003data}. We want to optimize over the cross-entropy loss induced by this dataset. We therefore sample an initial iterate $\theta_0$, an initial learning rate and an initial resolution for our optimizer, and launch an optimization run. We test against two fairly good optimizers for this task: tuned gradient descent and Newton descent. We use a fixed budget $k=10$ of dimensions we can sample at each iteration. The results for three different randomly generated dataset of different dimensions are presented in Figure \ref{fig::highd}. 
		
			The results presented here are consistent in our experiments - that is that our procedure competes with tuned optimizers that use respectively first and second order information. However, one major downside of our optimizer is clock-time performances - one optimization run in this simple set-up can take up to a minute for $d>50$, against a few seconds for gradient descent on the machine used for our experiments. Also, its performance is impacted by the under sampling that happens when the budget $k$ is strictly less than the dimension $d$. Increasing the budget $k$ improves the per-iteration performance but greatly impact the algorithm's clock-time (as the number of operations grows quadratically with $k$). In the following experiments, we propose sampling strategies for the pairs of dimensions used at every update that take advantage of the problem's structure to cope with this limitation.
		
%			\begin{figure}[h!]
%				\begin{center}
%					\includegraphics[width=1\linewidth]{highd_tests}
%					\caption{Tests runs for cross entropy loss of randomly generated binary classifications tasks, with budget $k=10$ of pair dimensions samples per iterations. The dimensionality of the problems are in order, from left to right: d=10, 20 and 50. Best viewed in color.}
%					\label{fig::highd}
%				\end{center}
%			\end{figure}
			
			\begin{figure}[h!]
			\centering
			\begin{subfigure}[b]{0.33\linewidth}
			{
				\centering
				\includegraphics[width=\textwidth]{bc_10}
				\caption{d=10}
				\label{fig::bc_10}
			}
			\end{subfigure}\hfill
			\begin{subfigure}[b]{0.33\linewidth}
			{
				\centering
				\includegraphics[width=\textwidth]{bc_20}
				\caption{d=20}
				\label{fig::bc_20}
			}
			\end{subfigure}\hfill
			\begin{subfigure}[b]{0.33\linewidth}
			{
				\centering
				\includegraphics[width=\textwidth]{bc_50}
				\caption{d=50}
				\label{fig::bc_50}
			}
			\end{subfigure}
			\caption[Test runs for cross entropy loss of randomly generated binary classifications]{Test runs for cross entropy loss of randomly generated binary classifications tasks of dimension $d$, with budget $k=10$ of pair samples per iterations. Best viewed in color.}
			\label{fig::highd}
		\end{figure}


		}
		
		\subsection{Small neural network}
		{
			\label{sec::net_exp}
			We now want to use Rover Descent for a more complicated task. We consider using a small neural network in order to solve the Iris dataset \cite{fisher1936use}, which consists of 150 instances of 4-dimensional inputs and their respectives labels (1,2, or 3). The neural network we use is a small neural network, with two hidden-layer of width 10 and a softmax output activation, alongside a cross-entropy loss. The dimension of its loss landscape is $d=193$. We compare our results with tuned gradient descent and Adam. 
			
			Figure \ref{fig::netk} show the results we obtained using the same sampling strategy as presented earlier (\emph{i.e} we sample uniformly at random $k$ dimensions for which we create all possible $k(k-1)/2$ pairs to create two-dimensional slices). It now appears that simply increasing $k$ is not enough to ensure good behavior. Because some dimensions are visited by the algorithm only late in the procedure, their corresponding learning rate is still equal to the initial one, which can lead to erratic behaviors close to local minimum. Also, increasing $k$ implies that for every dimension we sample, we increase the number of two-dimensional moves our algorithm receives. Unlike for the convex case of linear binary classification, it appears that taking their mean value is a poor strategy as it leads to a slight decrease in performance. 
			
			To improve those results, we decide to use a slightly different sampling strategy. For \emph{every} dimension, we are going to create pairs with $l$ other dimensions, sampled uniformly at random. The difference is that every dimension will be used at least once at every update. The number of pairs we create is now $l\times d$. Figure \ref{fig::netl} present the result obtained for different values of $l$ and proves the superiority of this approach over the previous one. 
			
			Finally, we decide to use the special structure of the neural network to improve our algorithm. We use the same sample strategy we just presented, except that now the $l$ dimensions needed for every dimensions are sampled within a pre-defined subset. More precisely, we want to leverage a block-diagonal structure of the Hessian of the neural network: for every dimension (which correspond to a weight or a bias of the neural network), we only create pairs with dimensions corresponding to a weight or bias belonging to the same layer. Figure \ref{fig::netlblock} present the results obtained, which again improves against the last one and compete with Adam on this task. Leveraging a block-diagonal approximation of the Hessian is not a new idea and was recently used to obtained state-of-the-art results on neural network optimization (\cite{martens2015optimizing},\cite{zhang2017block})
			
			\begin{figure}[h!]
			\centering
			\begin{subfigure}[b]{0.33\linewidth}
			{
				\centering
				\includegraphics[width=\textwidth]{iris_k}
				\caption{Pair sampling}
				\label{fig::netk}
			}
			\end{subfigure}\hfill
			\begin{subfigure}[b]{0.33\linewidth}
			{
				\centering
				\includegraphics[width=\textwidth]{iris_l}
				\caption{Per-dimension sampling}
				\label{fig::netl}
			}
			\end{subfigure}\hfill
			\begin{subfigure}[b]{0.33\linewidth}
			{
				\centering
				\includegraphics[width=\textwidth]{iris_lblock}
				\caption{Per-dimension block sampling}
				\label{fig::netlblock}
			}
			\end{subfigure}
			\caption[Test runs on a small neural network for Iris dataset]{Test runs on a small neural network for Iris dataset, with different sampling strategies for the two-dimensional slices. Best viewed in color.}
			\label{fig::net_test}
		\end{figure}

		}
	}


	
		\chapter{Future work}
	{
		\label{chap::future}
		
		\paragraph{} In a close future, we wish to work on making our algorithm scale, in order to test it on more complex model, such as the loss of deep neural networks. This might involve devising smarter, adaptative dimension sampling strategies. 
		
		We also wish to investigate three axis of research that we judge crucial: 
		\begin{itemize}
			\item it appears that the choice of the meta-dataset is essential to the meta-generalization abilities of the meta-learner. Answering questions about a sufficient meta-dataset for a given optimization task would be a first step. To achieve this goal, we can imagine performing ablation studies on the meta-dataset to scan for the features an optimizer would need to master the optimization of a complex, high-dimensional landscapes. Another possibility would be to set-up an adversarial setting to generate functions in the meta-dataset on which the current iteration of the meta-learner will perform poorly. 
			
			\item investigating the training method to optimize the meta-trainer itself also seems worthy of efforts. We namely want to investigate the links between meta-learning, hyper-parameter optimization and Bayesian reinforcement learning (see \cite{ghavamzadeh2015bayesian} for references). Indeed, we believe that a general framework for learning to optimize can be inspired by different fields. The crucial point is that, contrary to classical reinforcement learning or optimization problems, we wish to learn a good behaviors (mean or worst-case) over classes of problems and environments. 
			
			\item we also wish to study the links between the architecture of a meta-trainer and a given oracle. We indeed believe that particular structure (attention models, ..) could leverage crucial knowledge and learn efficient representation skills about a local landscape. For instance, building a graph-like structure over actively sampled function evaluations could help a meta-learner understand extremely complex landscapes and explore them more efficiently. 
		\end{itemize}

		
	}
	
	\chapter{Conclusion}
	{
		
		\paragraph{} In this report, we presented a framework for learning to optimize that build towards meta-generalization. By proposing to pose the problem as a navigation problem in high dimensional loss landscapes, using tools from reinforcement learning and supervised learning and providing a small but sufficient set of prototypical landscape that often arise in optimization problem, we learned an optimization algorithm that shows good behavior on a diverse class of held-out functions. 
		
		We provide some good results on a benchmark of hard two-dimensional functions, and we extend our approach to obtain some promising results in higher-dimensions - we give examples on a linear binary classification task and a shallow neural network multi-class loss. The results of this work will be published as a conference paper at LION 12 (LION: Learning and Intelligent Optimization) in June 2018. 
		
		We provide some ideas for future work, that will be tackled during a PhD starting shortly after this master thesis. 
		
		\paragraph{} I wish to thank Flavian Vasile, my supervisor at Criteo. It was a real pleasure working under his guidance for six months. I also wish to thank the whole Criteo research team, and especially Hadrien Hendrikx and Ugo Tanielian for all the interesting discussions we had while working together on each others current interests. 
	}
	
	\bibliographystyle{apalike}
 	 \bibliography{reportbib}
	 
	 \chapter{Appendix}
{
	\section{Landscape generation in details}
	{
		\label{sec::landscape_detail}
		
		We generate random instances of five landscapes: saddles, valleys, plateaus, cliffs and quadratic bowls. The quadratic bowls are generated by sampling random matrices $A$ and vectors $b$ and aggregating them in a quadratic loss $\lVert Ax - b \rVert_2^2$. The other four landscapes are generated by creating random gaussian fields with Mahalanobis-norm covariance functions. More precisely, we carefully sample a given number of points $x$ which are attributed a random value $f(x)$. We then sample a covariance function $k(x,x') = \frac{1}{2} x^T S^{-1} x$ with $S$ being a randomly generated definite positive matrix, carefuly set to generate the targeted landscape.We therefore create a class of functions $\mathcal{F}_{train}$, from which we can sample random instances of the different targeted landscapes. We also add randomness inside each of these instances by not always taking the mean of the generated random fields, but by sampling inside the resulting distribution over space.
		
		We hereinafter describe precisely how we generated samples of each modality in $\mathcal{F}_{train}$ (except quadratic bowls). This procedure is extremely similar to the one that is followed when one makes inference with Gaussian Processes - see \cite{rasmussen2006gaussian}. It starts by carefully sampling a collections of points $\mathcal{X} = \{x_1,\hdots,x_k\}$ and their associated values $\mathcal{V} = \{v_1,\hdots,v_k\}$ to create a specificly targeted landscape. We then sample a positive definite scaling matrix $S$ for the normalized Gaussian covariance function: 
		\begin{equation}
			k(x,x') = \frac{1}{\vert 2\pi S\vert^{1/2}}\exp{\left(-\frac{1}{2}x^TS^{-1}x'\right)}
		\end{equation} 
		Once these steps are completed, and for $F(x) \triangleq \left(f(x),v_1,\hdots,v_k\right)$, we make a Gaussian hypothesis over the joint distribution:
		\begin{equation}
			 p(F) = \mathcal{N}\left( F\, \vert\, 0 , \begin{pmatrix} K \\ k(x) \end{pmatrix} \right)
		\end{equation} 
		where $K = (k(x_i,x_j))_{i,j}$ and $k(x) = (k(x,x_i))_i$. 
		We then can evaluate the conditional distribution $p(f(x) \, \vert \, v_1,\hdots,v_k)$ (which is also a Gaussian) and sample from it to create a noisy version of $f(x)$. We repeat this procedure whenever we need to access to the value of one of the loss in $\mathcal{F}_{train}$ at a point $x$. 
		
		\paragraph{} The following lists details how $\mathcal{X}$ and $\mathcal{V}$ were sampled for each modalities of $\mathcal{F}_{train}$.
		\begin{itemize}
			\item \textbf{Valleys}: We set $\mathcal{X} = \{ 0_{\mathbb{R}^2} \}$ and sample $v_1$ uniformly at random in $[-5,0]$. We sample one value $\lambda_1$ in a positive truncated Gaussian distribution centered at 10 with variance 2. We then multiply it by a ratio $\rho$ uniformly sampled at random inside the interval $[100,200]$ to obtain $\lambda_2 = \rho \lambda_1$. We sample uniformly at random $\phi$ in $[0,2\pi]$, and create $S = R_\phi^T \text{diag}(\lambda_1,\lambda_2) R_\phi$ where $R_\phi$ is the two-dimensional rotation matrix of angle $\phi$. By that mean, we are able to create valleys of different width and orientation.
			\item \textbf{Saddles} : we sample $4$ points $x_1,\hdots, x_4$ uniformly at random within each quarter of the square $[-1,1]^2$ and place them in $\mathcal{X}$. We assign a random value to each one (sampled from a Gaussian distribution) so that to opposite points have values of similar signs. We then sample $\lambda_1,\lambda_2$ in a truncated positive Gaussian distribution centered at $10$ and with variance $2$. We sample a random angle $\phi$ and compute $S = R_\phi^T \text{diag}(\lambda_1,\lambda_2) R_\phi$.
			\item \textbf{Plateau+cliffs}:  $\mathcal{X} = \{ 0_{\mathbb{R}^2} \}$  and generate a single value $v_1$ from a positive truncated Gaussian distribution centered in -5 with variance $2$. We then create a matrix $S$ in the same fashion as for the previously described landscape. 
		\end{itemize}

	}
	
	\section{Policy evaluation}
	{
		\label{sec::policy_viz} 
%		
%		\begin{figure}[h!]
%			\centering
%			\begin{subfigure}[b]{0.5\linewidth}
%			{
%				\centering
%				\includegraphics[width=0.8\textwidth]{policy_seq_quad}
%				\caption{Quadratic}
%				\label{fig::policy_seq_quadratic}
%			}
%			\end{subfigure}\hfill
%			\begin{subfigure}[b]{0.5\linewidth}
%			{
%				\centering
%				\includegraphics[width=0.8\textwidth]{policy_seq_valleys}
%				\caption{Valley}
%				\label{fig::policy_seq_valley}
%			}
%			\end{subfigure}\\
%			\begin{subfigure}[b]{0.5\linewidth}
%			{
%				\centering
%				\includegraphics[width=0.8\textwidth]{policy_seq_saddle}
%				\caption{Saddle}
%				\label{fig::policy_seq_saddle}
%			}
%			\end{subfigure}\hfill
%			\begin{subfigure}[b]{0.5\linewidth}
%			{
%				\centering
%				\includegraphics[width=0.8\textwidth]{policy_seq_cliff}
%				\caption{Plateau+cliff}
%				\label{fig::policy_seq_cliff}
%			}
%			\end{subfigure}
%			\caption{Step-size and resolution trajectories for different instances of $\mathcal{F}_{train}$}
%			\label{fig::policy_seq}
%		\end{figure}

%		
%		This curves could suggest that the policy has learnt good scheduling schemes for each of the modalities in $\mathcal{F}_{train}$. We show here that this is not the case, by showing that the policy is able to adapt to sharp changes of its initial state. In Figure \ref{fig::policy_seq_ad}, we sample this initial state far from the distribution that was used at training time (typically, a learning rate that is 100 times smaller that it usually is at training time, and a resolution 100 bigger), for a quadratic bowl. We can see that the schedule adapts to these changes, and therefore that mostly changes in the local landscape seen by the agent affects its decision (and not a fixed schedule learnt by the RNN). 
%		
%		\begin{figure}[h!]
%			\centering
%			\includegraphics[width=0.4\linewidth]{policy_seq_ad}
%			\caption{Step-size and resolution can recover from a poorly set initial state}
%			\label{fig::policy_seq_ad}
%		\end{figure}	

		In Figure \ref{fig::policy_seq_contour}, we show optimization runs on contour plots of modalities of $\mathcal{F}_{train}$, with a visualization of the sample grid. 
		\begin{figure}
			\centering
			\begin{subfigure}[b]{0.4\linewidth}
			{
				\centering
				\includegraphics[width=0.8\textwidth]{fplot_quad}
				\caption{Quadratic}
			}
			\end{subfigure}
			\begin{subfigure}[b]{0.4\linewidth}
			{
				\centering
				\includegraphics[width=0.8\textwidth]{fplot_valley}
				\caption{Valley}
			}
			\end{subfigure}\\
			\begin{subfigure}[b]{0.4\linewidth}
			{
				\centering
				\includegraphics[width=0.8\textwidth]{fplot_saddle}
				\caption{Saddle}
				\label{fig::policy_seq_saddle}
			}
			\end{subfigure}
			\begin{subfigure}[b]{0.4\linewidth}
			{
				\centering
				\includegraphics[width=0.8\textwidth]{fplot_cliff}
				\caption{Plateau+cliff}
				\label{fig::policy_seq_cliff}
			}
			\end{subfigure}
			\caption{Contour plot of optimization runs for policy vizualisation on $\mathcal{F}_{train}$}
			\label{fig::policy_seq_contour}
		\end{figure}
		
	}
	
	
	\section{Precisions on $\mathcal{F}_{test}$}
	{
		\label{sec::ftest}
		We provide here some contour plots for the functions used in the meta-test dataset. We also provide their mathematical expression, as well as the position of their global optimum and the position of the initial iterate we used in our experiment. 
		
		\subsection*{Rosenbrock's function}
		{
			Rosenbrock's function's analytical expression is:
			\begin{equation}
				f(x) = 100(x_1-x_0^2)^2 + (x_0-1)^2
			\end{equation}
			and its contour plot is shown in \ref{fig::rosenbrock}.
			
			\begin{figure}[h!]
				\begin{center}
					\includegraphics[width=0.3\linewidth]{rosenbrock_ls}
					\caption[Contour plot of the Rosenbrock's function]{Contour plot of the Rosenbrock's function. The black circle indicates the position of the optimum and the blue square the position of the initial iterate.}
					\label{fig::rosenbrock}
				\end{center}
			\end{figure}
			
		}
		
		\subsection*{Ackley's function}
		{
			Acley's function's analytical expression is:
			\begin{equation}
				f(x) = -20\exp{\left(-0.2\sqrt{\frac{1}{2}(x_1^2+x_2^2)}\right)} - \exp{\left(\frac{1}{2}\cos{2\pi x_1} + \frac{1}{2}\cos{2\pi x_2}\right)} + 20 + e^1
			\end{equation}
			and its contour plot is shown in \ref{fig::ackley}.
			
			\begin{figure}[h!]
				\begin{center}
					\includegraphics[width=0.3\linewidth]{ackley_ls}
					\caption[Contour plot of the Ackley's function]{Contour plot of the Ackley's function. The black circle indicates the position of the optimum and the blue square the position of the initial iterate.}
					\label{fig::ackley}
				\end{center}
			\end{figure}
		}
		
		\subsection*{Rastrigin's function}
		{
			Rastrigin's function's analytical expression is:
			\begin{equation}
				f(x) = 20 + \sum_{i=1}^2 (x_i^2 - 10\cos{(2\pi x_i)})
			\end{equation}
			and its contour plot is shown in \ref{fig::Rastrigin}.
			
			\begin{figure}[h!]
				\begin{center}
					\includegraphics[width=0.3\linewidth]{rastrigin_ls}
					\caption[Contour plot of the Rastrigin function]{Contour plot of the Rastrigin function. The black circle indicates the position of the optimum and the blue square the position of the initial iterate.}
					\label{fig::Rastrigin}
				\end{center}
			\end{figure}
		}
		
		\subsection*{Maccornick's function}
		{
			Maccornick's function's analytical expression is:
			\begin{equation}
				f(x) = \sin{(x_1+x_2)} + (x_1-x_2)^2 -1.5x_1 + 2.5x_2 +1 
			\end{equation}
			and its contour plot is shown in \ref{fig::Maccornick}.
			
			\begin{figure}[h!]
				\begin{center}
					\includegraphics[width=0.3\linewidth]{mc_cornick_ls}
					\caption[Contour plot of the Rastrigin function]{Contour plot of the Maccornick function. The black circle indicates the position of the optimum and the blue square the position of the initial iterate.}
					\label{fig::Maccornick}
				\end{center}
			\end{figure}
		}
		
		\subsection*{Styblinski's function}
		{
			Styblinski's function's analytical expression is:
			\begin{equation}
				f(x) = \frac{1}{2} \sum_{i=1}^2 (x_i^4 - 16x_i^2+5x_i)
			\end{equation}
			and its contour plot is shown in \ref{fig::Styblinski}.
			
			\begin{figure}[h!]
				\begin{center}
					\includegraphics[width=0.3\linewidth]{styblinksi_ls}
					\caption[Contour plot of the Styblinski function]{Contour plot of the Styblinski function. The black circle indicates the position of the optimum and the blue square the position of the initial iterate.}
					\label{fig::Styblinski}
				\end{center}
			\end{figure}
		}
		
		\subsection*{Beale's function}
		{
			Beale's function's analytical expression is:
			\begin{equation}
				f(x) = (1.5-x_1+x_1x_2)^2 + (2.25-x_1+x_1x_2^2)^2 + (2.625-x_1+x_1x_2^3)^2
			\end{equation}
			and its contour plot is shown in \ref{fig::Beale}.
			
			\begin{figure}[h!]
				\begin{center}
					\includegraphics[width=0.3\linewidth]{beale_ls}
					\caption[Contour plot of the Beale function]{Contour plot of the Beale function. The black circle indicates the position of the optimum and the blue square the position of the initial iterate.}
					\label{fig::Beale}
				\end{center}
			\end{figure}
		}
	}
}

	
\end{document}